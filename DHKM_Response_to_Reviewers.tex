\documentclass[11pt]{article}
\usepackage{latexsym,amsfonts,amsmath,theorem,amssymb}

\newcommand{\il}{\left\langle}
\newcommand{\ir}{\right\rangle}
\newcommand{\bsb}{{\boldsymbol{b}}}
\newcommand{\bsc}{{\boldsymbol{c}}}
\newcommand{\bsgamma}{{\boldsymbol{\gamma}}}
\newcommand{\bseta}{{\boldsymbol{\eta}}}
\newcommand{\bstau}{{\boldsymbol{\tau}}}
\newcommand{\bsh}{{\boldsymbol{h}}}
\newcommand{\bsp}{{\boldsymbol{p}}}
\newcommand{\bsell}{{\boldsymbol{\ell}}}
\newcommand{\bst}{{\boldsymbol{t}}}
\newcommand{\bsr}{{\boldsymbol{r}}}
\newcommand{\bsu}{{\boldsymbol{u}}}
\newcommand{\bsv}{{\boldsymbol{v}}}
\newcommand{\bsx}{{\boldsymbol{x}}}
\newcommand{\bsq}{{\boldsymbol{q}}}
\newcommand{\bsy}{{\boldsymbol{y}}}
\newcommand{\bsz}{{\boldsymbol{z}}}
\newcommand{\bsw}{{\boldsymbol{w}}}
\newcommand{\bsS}{{\boldsymbol{S}}}
\newcommand{\bsW}{{\boldsymbol{W}}}
\newcommand{\bspitch}{{\boldsymbol{\,\pitchfork}}}
\newcommand{\bszero}{{\boldsymbol{0}}}
\newcommand{\bsone}{{\boldsymbol{1}}}
\newcommand{\rd}{\,\mathrm{d}}
\newcommand{\ri}{\mathrm{i}}
\newcommand{\re}{\mathrm{e}}
\newcommand{\bbR}{\mathbb{R}}
\newcommand{\reals}{\mathbb{R}}
\newcommand{\bbN}{\mathbb{N}}
\newcommand{\bbZ}{\mathbb{Z}}
\newcommand{\bbE}{\mathbb{E}}
\newcommand{\calA}{\mathcal{A}}
\newcommand{\calB}{\mathcal{B}}
\newcommand{\calC}{\mathcal{C}}
\newcommand{\calD}{\mathcal{D}}
\newcommand{\calJ}{\mathcal{J}}
\newcommand{\calK}{\mathcal{K}}
\newcommand{\calL}{\mathcal{L}}
\newcommand{\calZ}{\mathcal{Z}}
\newcommand{\calF}{\mathcal{F}}
\newcommand{\calG}{\mathcal{G}}
\newcommand{\calH}{\mathcal{H}}
\newcommand{\calI}{\mathcal{I}}
\newcommand{\calO}{\mathcal{O}}
\newcommand{\calQ}{\mathcal{Q}}
\newcommand{\calS}{\mathcal{S}}
\newcommand{\calW}{\mathcal{W}}
\newcommand{\tr}{{\tt T}}
\renewcommand{\pmod}[1]{\,(\bmod\,#1)}
\renewcommand{\mod}[1]{\,(\negthickspace\bmod #1)}
\newcommand{\ns}{\negthickspace\negthickspace}
\newcommand{\nns}{\negthickspace\negthickspace\negthickspace\negthickspace}
\newcommand{\mask}[1]{}
\newcommand{\esup}{\operatornamewithlimits{ess\,sup}}
\newcommand{\einf}{\operatornamewithlimits{ess\,inf}}
\newcommand{\e}{{\varepsilon}}
\newcommand{\setu}{{\mathfrak{u}}}
\newcommand{\setv}{{\mathfrak{v}}}
\newcommand{\setw}{{\mathfrak{w}}}
\newcommand{\setU}{{\mathfrak{U}}}
\newcommand{\sob}{{\rm Sob}}
\newcommand{\norm}[1]{\left\Vert#1\right\Vert}
\newcommand{\abs}[1]{\left\vert#1\right\vert}
\newcommand{\sset}{[s]}
\newcommand{\To}{\rightarrow}
\newcommand{\dimtr}{{\rm dim}^{\rm trnc}}
\newcommand{\dimtrint}{{\rm dim}^{\rm trnc-int}}

\begin{document}

\begin{center}
 
{\Large Response to the Reviewers of the Manuscript\\
``Adaptive Approximation for Multivariate Linear Problems with Inputs Lying in a Cone''}
\end{center}

\medskip

\noindent We would like to thank you for your careful reviews of our manuscript and your helpful comments. 

Following the suggestions of one of the reviewers, we have restructured the introductory parts of the paper. Thereby, we address several of the remarks 
made by both referees. Relevant changes in the paper are visible by red text. 

Below, we respond to the points made in your reports.

\subsection*{Referee Report 1 ``Summary. The authors study \ldots '' (2 pages):}

\begin{itemize}
\item Remarks (1)--(4): done. 
\item Remark (5) We have rewritten the first and second sections extensively.
\item Remark (6): What we mean by "this example" is the example of functions defined over $[-1,1]^d$ with a Chebyshev polynomial basis. We reformulated this passage to make this point clearer. 
\item Remarks (7), (8): Due to the new structure of the paper, we hope to have addressed these issues accordingly. 
\item Remark (9): done. 
\item Remark (10): Due to the new structure of the paper, we hope to have addressed these issues accordingly. 
\item Remarks (11), (12): Due to the new structure of the paper, we hope to have addressed these issues accordingly. 
\item Remark (13): We put this definition into a separate paragraph. 
\item Remark (14): done. 
\item Remark (15): Due to the new structure of the paper, we hope to have addressed these issues accordingly. 
\item Remark (16): done. 
\item Remark (17): The motivation for defining this cone is stated in the beginning of the section, but we have reformulated this passage to make it clearer. 
\item Remark (18): Thanks, we have provided further details in Algorithm 4 for the step ``Compute data-driven POSD weights'', as well as for the sentence around (41). With this, the adaptive procedure is better defined, and we believe it should be called an algorithm.
\end{itemize}

\subsection*{Referee Report 2 ``The paper studies \ldots'' (4 pages):}

\begin{itemize}
\item Major issues 1.a)--1.d), 2.a), 2.b): Done. We have modified the introduction, and re-organized it into two sections. We have added a statement after Corollary 1 regarding a result for the example with respect to the smoothness parameter $r$
\item Remark (i): The ordering is not necessarily unique, as we might have weights 
that are equal. 
\item Remark (ii): the referee is correct. Thanks for pointing this out. 
\item Remark (iii): done. 
\item Remark (iv): The paper will be a chapter in a proceedings volume.
\item Remark (v): The re-structuring of the paper should have resolved this issue.
\item Remark (vi): done. 
\item Remark (vii): We changed ``Choosing" to ``Identifying".
\item Remark (viii): Theoretically, we could simplify this as suggested by the reviewer; however, we would like to keep the influence of $\varepsilon$ and $R$ separate and explicit. 
\item Remark (ix): We have kept this result as a theorem, but due to the re-structuring of the paper, it is no longer part of the introduction. 
\item Remark (x): We have moved the contents of the previous Section 1.5 as suggested by the reviewer. 
\item Remark (xi): The cone in equation (16) is now denoted as $\calC_{\boldsymbol{\lambda},n_1,A}$. 
\item Remark (xii): We have fixed the notation in old Proposition 1, which is no longer a proposition. Thanks for pointing this out. We have also added a new proposition  to discuss the relationship between our cone $\mathcal{C}_{\boldsymbol{\lambda},n_1,A}$ and a new cone $\mathcal{C}'_{\boldsymbol{\lambda},\boldsymbol{\omega},A'}$.
\item Remark (xiii): We have added the results with smoothness parameters $r$ in the new Corollary 10.
\item Remark (xiv): This portion has been re-written. 
\item Remark (xv): The cone in equation (28) is now denoted as $\calC_{\boldsymbol{\lambda},\boldsymbol{n},a,b}$.  Also, we have added extensive discussion in Section 4.2 related to the illustrative example.  See Theorem 15.
\item Remark (xvi): Done.
\item Remark (xvii): Some wording has been added.
\item Remark (xviii): We have added more elaboration in the new Sections 3 and 4 to apply our results to the illustrative example.
\item Remark (xix): Here, we mean a ``small number of coordinates in $f$ are important''. We have made this change in the paper.
\item Remark (xx): These guiding principles help guide the inference of weights $\boldsymbol{\lambda}$ (and its corresponding cone $\calC_{\boldsymbol{\lambda},n_1,A}$) from initial data. We have added a sentence on this on page 29. 
\item Remark (xxi): Thanks, the cone in equation (16) is now denoted as $\calC_{\boldsymbol{\lambda},n_1,A}$, and the data-inferred cone on page 30 is now denoted as $\bar{\calC}_{n_1,A} = \calC_{\bar{\boldsymbol{\lambda}},n_1,A}$.
\item Remark (xxii): Yes, this is correct. We have added a sentence after equation (40) clarifying this.
\item Remark (xxiii): The data-adaptive cone $\bar{\calC}_{n_1,A}$ is the earlier cone $\calC_{\boldsymbol{\lambda},n_1,A}$ in equation (16) with data-adaptive weights $\boldsymbol{\lambda} = \bar{\boldsymbol{\lambda}}$. We have clarified this in equation (42).

\end{itemize}

\bigskip 

\bigskip

\noindent With kind regards,\\[0.25cm]
The authors, 

\noindent Y. Ding, F.J. Hickernell, P. Kritzer, and S. Mak.


\end{document}
