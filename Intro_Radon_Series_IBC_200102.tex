\documentclass[11pt,a4paper]{article}
\usepackage{latexsym,amsfonts,amsmath,graphics}
\usepackage{epsfig}
\usepackage[notref,notcite]{showkeys}
%\usepackage[active]{srcltx}
\usepackage{enumitem}
\usepackage{mathrsfs}
\usepackage[usenames,dvipsnames,svgnames,table]{xcolor}

\setlength{\textheight}{24cm} \setlength{\textwidth}{16cm}
\setlength{\hoffset}{-1.3cm} \setlength{\voffset}{-1.8cm}
\newtheorem{theorem}{Theorem}
\newtheorem{lemma}{Lemma}
\newtheorem{corollary}{Corollary}
\newtheorem{proposition}{Proposition}
\newtheorem{conjecture}{Conjecture}
\newtheorem{definition}{Definition}
\newtheorem{algorithm}{Algorithm}
\newtheorem{remark}{Remark}
\newenvironment{proof}{\begin{trivlist}
\item[\hskip\labelsep{\it Proof.}]}{$\hfill\Box$\end{trivlist}}

\newcommand{\norm}[1]{\left\Vert#1\right\Vert}
\newcommand{\To}{\rightarrow}
\newcommand{\Ref}[1] {\textup{(\ref{#1})}}
\newcommand{\satop}[2]{\stackrel{\scriptstyle{#1}}{\scriptstyle{#2}}}
\newcommand{\sump}{\mathop{{\sum}'}}
\newcommand{\bsDelta}{\boldsymbol{\Delta}}
\newcommand{\bsa}{\boldsymbol{a}}
\newcommand{\bsbeta}{\boldsymbol{\beta}}
\newcommand{\bsgamma}{\boldsymbol{\gamma}}
\newcommand{\bseta}{\boldsymbol{\eta}}
\newcommand{\bsalpha}{\boldsymbol{\alpha}}
\newcommand{\bslambda}{\boldsymbol{\lambda}}
\newcommand{\bsnu}{\boldsymbol{\nu}}
\newcommand{\bsmu}{\boldsymbol{\mu}}
\newcommand{\bsk}{\boldsymbol{k}}
\newcommand{\bsl}{\boldsymbol{l}}
\newcommand{\bsv}{\boldsymbol{v}}
\newcommand{\bsw}{\boldsymbol{w}}
\newcommand{\bsb}{\boldsymbol{b}}
\newcommand{\bsx}{\boldsymbol{x}}
\newcommand{\bsh}{\boldsymbol{h}}
\newcommand{\bsg}{\boldsymbol{g}}
\newcommand{\bst}{\boldsymbol{t}}
\newcommand{\bsz}{\boldsymbol{z}}
\newcommand{\bsell}{\boldsymbol{\ell}}
\newcommand{\bsu}{\boldsymbol{u}}
\newcommand{\bsone}{\boldsymbol{1}}
\newcommand{\tru}{{\rm tr}}
\newcommand{\maxp}{\operatornamewithlimits{max\phantom{p}}}
\newcommand{\infp}{\operatornamewithlimits{inf\phantom{p}}}
\newcommand{\bsy}{\boldsymbol{y}}
\newcommand{\bssigma}{\boldsymbol{\sigma}}
\newcommand{\D}{{\cal D}}
\newcommand{\E}{{\cal E}}
\newcommand{\calK}{{\cal K}}
\newcommand{\cP}{{\cal P}}
\newcommand{\cG}{{\cal G}}
\newcommand{\cQ}{{\cal Q}}
\newcommand{\e}{{\varepsilon}}
\newcommand{\wal}{{\rm wal}}
\newcommand{\sob}{{\rm sob}}
\newcommand{\cost}{{\rm cost}}
\newcommand{\comp}{{\rm comp}}
\newcommand{\dsh}{{\rm dsh}}
\newcommand\setu{{\mathfrak{u}}}
\newcommand{\uu}{\mathfrak{u}}
\newcommand{\de}{{\rm e}}
\newcommand{\swal}{{\rm wal}}
\newcommand{\landau}{{\cal O}}
\newcommand{\mc}{{\rm mc}}
\newcommand{\icomp}{\mathtt{i}}
\newcommand{\bszero}{\boldsymbol{0}}
\newcommand{\rd}{\,\mathrm{d}}
\newcommand{\NN}{\mathbb{N}}
\newcommand{\ZZ}{\mathbb{Z}}
\newcommand{\integer}{\ZZ}
\newcommand{\RR}{\mathbb{R}}
\newcommand{\CC}{\mathbb{C}}
\newcommand{\FF}{\mathbb{F}}
\newcommand{\QQ}{\mathbb{Q}}
\newcommand{\calH}{\mathcal{H}}
\newcommand{\LL}{\mathcal{L}_{\bsg}}
\newcommand{\LLu}{\mathcal{L}_{\setu,\bsg}}
\newcommand{\qed} {\hfill \Box \vspace{0.5cm}}
\renewcommand{\pmod}[1]{\,(\bmod\,#1)}
\newcommand{\ns}{\negthickspace\negthickspace}
\newcommand{\nns}{\negthickspace\negthickspace\negthickspace\negthickspace}
\newcommand{\MC}{{\rm MC}_{n,s}}
\newcommand{\EE}{\mathbb{E}}
\newcommand{\il}{\left<}
\newcommand{\ir}{\right>}
\def\qed{\hfill$\Box$}
\newcommand{\abs}[1]{\left\vert#1\right\vert}
\def\calV{{\cal V}}


\DeclareMathOperator*{\esssup}{ess\,sup}

\allowdisplaybreaks


\definecolor{darkmagenta}{RGB}{139,0,139}
\definecolor{darkorange}{RGB}{180,60,0}

\newcommand{\fred}[1]{{\color{darkorange}{#1}}}
\newcommand{\peter}[1]{{\color{darkmagenta}{#1}}}

\begin{document}



 \title{Preface: Multivariate Algorithms and Information-Based Complexity}
% 
% \author{\ }
% 
 \maketitle
% 
% 
% 
% \begin{abstract}
% Abstract. 
% \end{abstract}

This book comprises several articles authored or co-authored by invited speakers in the workshop ``Multivariate Algorithms 
and Information-Based Complexity'' which was part of the 
RICAM Special Semester on ``Multivariate Algorithms and their Foundations in Number Theory'' in the fall of 2018. 
The Special Semester consisted of four larger and two smaller workshops on various topics ranging from 
Pseudo-Randomness and Discrepancy Theory to Information-Based Complexity and Uncertainty Quantification. The present book 
corresponds to the second workshop of the program, which
took place at the Johann Radon Institute for Computational and Applied Mathematics (RICAM) of the Austrian Academy 
of Sciences in Linz, Austria, on November 5--9, 2018. 

The workshop and the papers contained in this book focus on 
multivariate continuous problems occurring in a multitude of practical applications, such as physics,
finance, computer graphics, and chemistry. In many cases, the number of variables in the problem
can be in the hundreds or thousands. The information complexity of a given problem is the minimal
number of information operations that is needed to solve the problem within a certain error
threshold. Typical examples of information operations are, e.g., function values or linear functionals.
The field of Information-Based Complexity (IBC), founded by Traub and Wozniakowski in the 1980s,
deals with the analysis of the information complexity for multivariate problems, and, in particular,
with the question how it depends on the number of variables and the error threshold. A crucial
question is, for instance, under which circumstances one can avoid a curse of dimensionality. This
book addresses the analysis of multivariate (continuous) problems, and in particular this shall be
done from the viewpoint of IBC.

The chapter ``The Control Variate Integration Algorithm for Multivariate Functions
Defined at Scattered Data Points'' studies a method of approximating the integral 
of a multivariate function in which one uses the exact integral of a control variate based on 
a least-squares multivariate quasi-interpolant. Numerical examples demonstrate that 
such an algorithm can help to deal with the curse of dimensionality for multivariate least-squares 
fits. The second chapter, titled ``An Adaptive Random Bit Multilevel
Algorithm for SDEs'', considers the approximations of expectations for functionals applied 
to the solutions of stochastic differential equations, by employing Monte Carlo methods 
based on random bits instead of random numbers; an adaptive random bit multilevel algorithm is provided 
and compared numerically to other methods. The chapter ``RBF-Based Penalized Least-Squares
Approximation of Noisy Scattered Data on the Sphere'' deals with the approximation of 
noisy scattered data on the 2-dimensional unit sphere, in particular global and local  
penalized least-squares approximation based on radial basis functions (RBFs). 
The authors of the fourth chapter in the book, ``On the Power of Random Information'', 
consider a problem from the core of IBC theory, namely the quality of random information 
in approximation and integration problems. Random information is compared to optimal 
information for such problems, and it is shown that in some cases random information 
can be almost optimal, whereas in other cases it may perform yield much worse 
results than optimal information. The chapter ``Optimality Criteria for Probabilistic Numerical Methods'' 
is in the field of Bayesian decision theory; to be more precise, the authors study 
an optimality criterion from Bayesian experimental design and considers its implied 
optimal information in the numerical context. Furthermore, the chapter compares this information 
to information commonly used in average-case-optimal numerical methods. The sixth chapter of the 
book, ``$\varepsilon$-Superposition and Truncation Dimensions, and Multivariate Decomposition Method
for $\infty$-Variate Linear Problems'' deals with linear problems on weighted normed function spaces, 
the elements of which depend on infinitely many variables. The author on so-called $\varepsilon$-truncation 
and $\varepsilon$-superposition dimension, as well as multivariate decomposition methods, which 
are means to reduce the number of variables in the problem to a relatively small finite number; 
anchored decomposition and also other types of decomposition (like, e.g., ANOVA decomposition) are used. 
The chapter ``Adaptive Approximation for Multivariate Linear Problems with Inputs Lying in a Cone'' is 
the final chapter of the book, and considers adaptive approximation algorithms for functions lying in particular subsets 
of certain function spaces. Contrary to many papers on IBC it is not assumed that the functions to be 
approximated do not lie in a ball of a certain radius, but instead in a cone. It is shown that for such problems 
it is an advantage to use adaptive instead of non-adaptive algorithms for function approximation. 







All chapters were reviewed by internationally renowoned experts in this field. We wish to thank the anonymous referees 
for their precious help.

We would like to express our gratitude towards all speakers of the 
workshop for giving excellent talks on their respective fields of expertise and for contributing to the success of 
the meeting: 
\begin{itemize}
\item Paul Constantine (University of Colorado), 
\item Martin Ehler (University of Vienna), 
\item Mario Hefter (TU Kaiserslautern), 
\item Stefan Heinrich (TU Kaiserslautern), 
\item Kerstin Hesse (University of Paderborn),
\item Fred J.~Hickernell (Illinois Institute of Technology), 
\item James (Mac) Hyman (Tulane University),
\item Thomas M\"{u}ller-Gronbach (University of Passau), 
\item Erich Novak (University of Jena), 
\item Chris Oates (Newcastle University and Alan Turing Institute), 
\item Houman Owhadi (California Institute of Technology),
\item Sergei Pereverzyev (RICAM), Leszek Plaskota (University of Warsaw), 
\item Pawe\l{} Przyby\l{}owicz (AGH University of Science and Technology Krak\'{o}w),
\item Klaus Ritter (TU Kaiserslautern), 
\item Ingo Steinwart (University of Stuttgart),
\item Michaela Sz\"{o}lgyenyi (University of Klagenfurt), 
\item Greg W.~Wasilkowski (University of Kentucky), 
\item Henryk Wo\'{z}niakowski (Columbia University and University of Warsaw),
\item Henry Wynn (London School of Economics), 
\item Larisa Yaroslavtseva (University of Passau).
\end{itemize}


We also would like to thank all participants attending the workshop, 
and Annette Weihs and Wolfgang Forsthuber for administrative and technical support. Further information 
on the RICAM Special Semester ``Multivariate Algorithms and their Foundations in Number Theory'' can 
be found at
\begin{center}
 https://www.ricam.oeaw.ac.at/specsem/specsem2018/ 
\end{center}




The financial support of the Johann Radon Institute for Computational and Applied Mathematics (RICAM) 
of the Austrian Academy of Sciences is gratefully acknowledged.


\bigskip

Fred J.~Hickernell and Peter Kritzer,

Chicago and Linz, January 2020.


\end{document}
