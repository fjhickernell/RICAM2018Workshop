\documentclass[11pt]{article}
\usepackage{latexsym,amsfonts,amsmath,theorem,color,amssymb}

\newcommand{\il}{\left\langle}
\newcommand{\ir}{\right\rangle}
\newcommand{\bsb}{{\boldsymbol{b}}}
\newcommand{\bsc}{{\boldsymbol{c}}}
\newcommand{\bsgamma}{{\boldsymbol{\gamma}}}
\newcommand{\bseta}{{\boldsymbol{\eta}}}
\newcommand{\bstau}{{\boldsymbol{\tau}}}
\newcommand{\bsh}{{\boldsymbol{h}}}
\newcommand{\bsp}{{\boldsymbol{p}}}
\newcommand{\bsell}{{\boldsymbol{\ell}}}
\newcommand{\bst}{{\boldsymbol{t}}}
\newcommand{\bsr}{{\boldsymbol{r}}}
\newcommand{\bsu}{{\boldsymbol{u}}}
\newcommand{\bsv}{{\boldsymbol{v}}}
\newcommand{\bsx}{{\boldsymbol{x}}}
\newcommand{\bsq}{{\boldsymbol{q}}}
\newcommand{\bsy}{{\boldsymbol{y}}}
\newcommand{\bsz}{{\boldsymbol{z}}}
\newcommand{\bsw}{{\boldsymbol{w}}}
\newcommand{\bsS}{{\boldsymbol{S}}}
\newcommand{\bsW}{{\boldsymbol{W}}}
\newcommand{\bspitch}{{\boldsymbol{\,\pitchfork}}}
\newcommand{\bszero}{{\boldsymbol{0}}}
\newcommand{\bsone}{{\boldsymbol{1}}}
\newcommand{\rd}{\,\mathrm{d}}
\newcommand{\ri}{\mathrm{i}}
\newcommand{\re}{\mathrm{e}}
\newcommand{\bbR}{\mathbb{R}}
\newcommand{\reals}{\mathbb{R}}
\newcommand{\bbN}{\mathbb{N}}
\newcommand{\bbZ}{\mathbb{Z}}
\newcommand{\bbE}{\mathbb{E}}
\newcommand{\calA}{\mathcal{A}}
\newcommand{\calB}{\mathcal{B}}
\newcommand{\calC}{\mathcal{C}}
\newcommand{\calD}{\mathcal{D}}
\newcommand{\calJ}{\mathcal{J}}
\newcommand{\calK}{\mathcal{K}}
\newcommand{\calL}{\mathcal{L}}
\newcommand{\calZ}{\mathcal{Z}}
\newcommand{\calF}{\mathcal{F}}
\newcommand{\calG}{\mathcal{G}}
\newcommand{\calH}{\mathcal{H}}
\newcommand{\calI}{\mathcal{I}}
\newcommand{\calO}{\mathcal{O}}
\newcommand{\calQ}{\mathcal{Q}}
\newcommand{\calS}{\mathcal{S}}
\newcommand{\calW}{\mathcal{W}}
\newcommand{\tr}{{\tt T}}
\renewcommand{\pmod}[1]{\,(\bmod\,#1)}
\renewcommand{\mod}[1]{\,(\negthickspace\bmod #1)}
\newcommand{\ns}{\negthickspace\negthickspace}
\newcommand{\nns}{\negthickspace\negthickspace\negthickspace\negthickspace}
\newcommand{\mask}[1]{}
\newcommand{\esup}{\operatornamewithlimits{ess\,sup}}
\newcommand{\einf}{\operatornamewithlimits{ess\,inf}}
\newcommand{\e}{{\varepsilon}}
\newcommand{\setu}{{\mathfrak{u}}}
\newcommand{\setv}{{\mathfrak{v}}}
\newcommand{\setw}{{\mathfrak{w}}}
\newcommand{\setU}{{\mathfrak{U}}}
\newcommand{\sob}{{\rm Sob}}
\newcommand{\norm}[1]{\left\Vert#1\right\Vert}
\newcommand{\abs}[1]{\left\vert#1\right\vert}
\newcommand{\sset}{[s]}
\newcommand{\To}{\rightarrow}
\newcommand{\dimtr}{{\rm dim}^{\rm trnc}}
\newcommand{\dimtrint}{{\rm dim}^{\rm trnc-int}}
\newcommand{\Response}[1]{{\color{blue}#1}}
\thispagestyle{empty}

\textwidth 6.5in
\hoffset -0.75in

\begin{document}

\begin{center}
 
{\Large Response to the Reviewers of the Manuscript\\
``Adaptive Approximation for Multivariate Linear Problems with Inputs Lying in a Cone''\quad
Revision 1}
\end{center}

\medskip

\noindent Thank you for your careful reviews and comments on our revised manuscript. We have revised our manuscript in light of these remarks.

\subsection*{Referee Report ``The manuscript is improved \ldots}

\begin{itemize}

\item page 3, line 2: ``it is less restrictive than only being able to
sample function values". \ldots  \Response{Change has been made.}

\item page 3, ``We call an algorithm non-adaptive if". \ldots \Response{Clarification has been added.}

\item Formula (26):  It seems to me that 11by (25)" should not be part
of formula (26) \ldots  \Response{This has been changed in some places, but has been left as is for those multi-line equations where we wanted to refer to precisely to what equation was supporting what part of the argument.}
\end{itemize}



\subsection*{Referee Report ``The paper has significantly improved.  \ldots'':}

\begin{enumerate}

\renewcommand{\labelenumi}{\alph{enumi})} 
\item Page 2, line -3, in the formula it should be \ldots \Response{Done.}

\item  Page 3, line 21, the notation for the computational complexity \ldots  \Response{Done.}

\item  Page 4, line 21, it should be “precise definition” \ldots \Response{Yes!  Corrected.}


\item Section 2: After my remark in the first report that the setting is actually a problem of diagonal operators between sequence spaces \ldots \Response{Pietsch 1978 is now referred to before Lemma 1 and again at the beginning of Section 3.2.}

\item Page 7: The norm $||\cdot ||_{\mathcal{G}}$ should be given earlier \ldots \Response{Done.}

\item Page 22: In formula (30) \ldots \Response{Yes, thank you.  Corrected.}

\item  Page 27, to me the shape of the blocks \ldots \Response{You are correct for the illustrative example.  A sentence stating this has been added.}

\end{enumerate}

\bigskip 

\bigskip

\noindent With kind regards,\\[0.25cm]
The authors, 

\noindent Y. Ding, F.J. Hickernell, P. Kritzer, and S. Mak


\end{document}
